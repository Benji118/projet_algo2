\documentclass[11pt,english]{article}
\usepackage[T1]{fontenc}
\usepackage{babel}
\usepackage[utf8]{inputenc}
\usepackage{fancyhdr}
\usepackage{lastpage}
\usepackage{graphicx}
\usepackage[T1]{fontenc}
\usepackage{amsmath,amssymb}
\usepackage{fullpage}
\usepackage{url}
\usepackage{xspace}
\usepackage{listings}
\lstset{
morekeywords={abort,abs,accept,access,all,and,array,at,begin,body,
case,constant,declare,delay,delta,digits,do,else,elsif,end,entry,
exception,exit,for,function,generic,goto,if,in,is,limited,loop,
mod,new,not,null,of,or,others,out,package,pragma,private,
procedure,raise,range,record,rem,renames,return,reverse,select,
separate,subtype,task,terminate,then,type,use,when,while,with,
xor,abstract,aliased,protected,requeue,tagged,until},
sensitive=f,
morecomment=[l]--,
morestring=[d]",
showstringspaces=false,
basicstyle=\small\ttfamily,
keywordstyle=\bf\small,
commentstyle=\itshape,
stringstyle=\sf,
extendedchars=true,
columns=[c]fixed
}
% CI-DESSOUS: conversion des caractères accentués UTF-8
% en caractères TeX dans les listings...
\lstset{
literate=%
{À}{{\`A}}1 {Â}{{\^A}}1 {Ç}{{\c{C}}}1%
{à}{{\`a}}1 {â}{{\^a}}1 {ç}{{\c{c}}}1%
{É}{{\'E}}1 {È}{{\`E}}1 {Ê}{{\^E}}1 {Ë}{{\"E}}1%
{é}{{\'e}}1 {è}{{\`e}}1 {ê}{{\^e}}1 {ë}{{\"e}}1%
{Ï}{{\"I}}1 {Î}{{\^I}}1 {Ô}{{\^O}}1%
{ï}{{\"i}}1 {î}{{\^i}}1 {ô}{{\^o}}1%
{Ù}{{\`U}}1 {Û}{{\^U}}1 {Ü}{{\"U}}1%
{ù}{{\`u}}1 {û}{{\^u}}1 {ü}{{\"u}}1%
}
%%%%%%%%%%
% TAILLE DES PAGES (A4 serré)
\setlength{\parindent}{0pt}
\setlength{\parskip}{1ex}
\setlength{\textwidth}{17cm}
\setlength{\textheight}{24cm}
\setlength{\oddsidemargin}{-.7cm}
\setlength{\evensidemargin}{-.7cm}
\setlength{\topmargin}{-.5in}
%%%%%%%%%%
% EN-TÊTES ET PIED DE PAGES
%% \pagestyle{fancyplain}
\renewcommand{\headrulewidth}{0pt}
\addtolength{\headheight}{1.6pt}
\addtolength{\headheight}{2.6pt}
\lfoot{}
\cfoot{\footnotesize\sf TPL Algo - Nœuds}
\rfoot{\footnotesize\sf page~\thepage/\pageref{LastPage}}
\lhead{}
%%%%%%%%%%
% COMMANDES PERSONNALISEES
\newcommand{\shellcmd}[1]{\\\indent\indent\texttt{\footnotesize\# #1}\\}
%%%%%%%%%

\author{
{Felix Hanleim}
\and
{Benjamin Lebit}
}
\title{Projet 1 - Algorithmique et Structures de Donnees 2}

\begin{document}
\maketitle
\section{Utilisation}
Notre programme exécute bien l'algorithme de décomposition d'un polynôme 
non monotone en polynômes monotones comme décrit dans l'énoncé. 
Pour l'exécution du programme, il faudra utiliser la commande :
\shellcmd{./main num.in output.svg}
avec num un entier de 1 à 8 correspondant aux 8 fichiers d'entrée fournit dans 
le sujet et ouput.svg le fichier svg de sortie.
On compilera au préalable avec :
\shellcmd{gnatmake main.adb}


\section{Organisation du code et structures de données}
\subsection{Paquets}
Le programme est décomposé en 6 paquets différents :
\begin{enumerate}
\item \emph{main} : Le main comprenant l'algorithme principal implémenté par 
l'appel des autres modules et utilisant les structures de données du paquet objets.
\item \emph{objets} : Comprend les définitions de tous les objets et structures
 de données utilisées dans le projet, ainsi que les fonctions de comparaisons utiles
 à l'algorithme principal.
\item \emph{abr} : Gère l'implémentation des arbres binaires de recherche ainsi 
que les procédures compte\_position et noeuds\_voisins imposées par l'énoncé.
\item \emph{parseur} : Permet la lecture et la sauvegarde des données du fichier 
in d'entrée ainsi que la modification des coordonnées utilisés lors du tracé svg.
\item \emph{geometry} : contient les procédures permettant de qualifié un segment 
comme entrant ou sortant et de raccordement des segments verticaux de séparation 
aux côtés du polygône. 
\item \emph{svg} : Contient les procédures d'écriture en svg nécessaires au tracé 
des polygônes en entrée ainsi que des segments verticaux de séparation de l'algorithme.
\end{enumerate}
\subsection{Structures}
La structure choisie de sauvegarde des données du fichier .in est un tableau 
d'enregistrement d'un sommet du polygône et de ses 2 segments qui permet un 
parcours facile lors de l'algorithme principale. La structure choisie pour 
l'arbre binaire est le même que celle proposée par l'énoncé.
\end{document}